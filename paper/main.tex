\documentclass[a4paper,10pt]{article}
\usepackage[british]{babel}
\usepackage[T1]{fontenc}
\usepackage[hidelinks]{hyperref}
\usepackage[a4paper]{geometry}

\usepackage{tabularx}
\usepackage{graphicx}
\usepackage{rotating}
\usepackage{float}
\usepackage{xspace}

% Math packages
\usepackage[sc]{mathpazo}
\usepackage{amsthm}
\usepackage{algorithmic}
\usepackage{amsmath}

\usepackage{listings}

% Document properties
\title{Welch-Lynch clock synchronization protocol \\\textit{Modelling in UPPAAL}}
\author{
	Wouter Geraedts \\ \small{\texttt{w.geraedts@student.ru.nl}} \and
	Ko Stoffelen     \\ \small{\texttt{kostoffelen@student.ru.nl}}
}
\date{}
\linespread{1.05}

% Terminology
\newcommand{\UPPAAL}{UPPAAL\xspace}

\begin{document}
	\maketitle

%%% Contents %%%
%Originele paper

%Zoeken naar concrete constanten

%Aanpassingen voor snelheid
%	Gebruik van C-functies [step1]
%	Gebruik van select-statements [step2]
%	Gebruik van local_time uit originele paper [step3]
%	Gebruik van SUM uit originele paper [step3minimized]
%	Timejumps [step3minimized]
%	Scalarset reduction [step4scalar]
%	Uitgebreide guards (forall i.p.v. local var C) [step2]

%Aanpassingen voor generiekerheid
%	Diff variable upgrade
%	Concrete types
%	Min en max (zelfs trager dan SUM) [step4]

%Resultaten: Vergelijkingen in performance
%Faulty process, niet mogelijk
%	Werk om te doen om het alsnog te maken (array voor min en max; zorgen dat het kan met scalarset (id_t))

\section{Introduction}

%Over dit paper
%Relatie met originele paper
%Opbouw van dit paper
%Korte verwijzing naar conclusie

This report is part of an assignment for the Analysis of Embedded Systems course in 2012--2013. Our goal was to model and analyze the Welch-Lynch fault-tolerant clock synchronization protocol \cite{Welch1984Anew} using \UPPAAL. A previous attempt has been made in 2004 \cite{Aceto2004Notes}, when the model had to be greatly simplified. New \UPPAAL features like symmetry reduction, a richer syntax and a more powerful verification engine in general, make that the model-based analysis deserved another chance.

The algorithm in its most general form still does not directly translate to an \UPPAAL model that is actually verifiable. This report first describes what optimizations, simplifications and updates have been applied to a general model. Then in section~\ref{sec:faulty} it is explained what else would have been needed to be able to verify a general version of the protocol. In section~\ref{sec:remarks} some remarks are made on \cite{Aceto2004Notes}, which explain some troubles that appeared in the process of this analysis. Finally in section~\ref{sec:performance} the performance of our optimized model is compared to the 2004 version, after which we draw our conclusions. \UPPAAL is able to verify the correctness of the protocol for three participating processes with a highly optimized model, but then we are already approaching the limit of \UPPAAL{}s capabilities.

\section{Optimizations}

%Aanpassingen voor snelheid
%Toelichten dat het leesbaarder is a.d.v. A First Introduction to UPPAAL

\section{Towards faulty process\label{sec:faulty}}

%Aanpassingen voor generiekerheid
%Aanpassingen die we nog wilde doen, indien we 5 processen hadden kunnen modelleren

\section{Remarks on \cite{Aceto2004Notes}\label{sec:remarks}}

%Niet duidelijk of ze daadwerkelijk het model hebben kunnen doorrekenen
%Geen concrete waarden voor constanten; beschrijving van zoektocht
%Onduidelijk wat het type is van variablen

\section{Performance\label{sec:performance}}

%Vergelijking tussen verschillende versies van ons, en uiteindelijke versie van aceto2004notes

\section{Conclusion}
	
%We kunnen succesvol 3 processen door laten rekenen, sneller dan toen (40 sec vs. <TODO>)
%Als 4 haalbaar blijkt te zijn, dan dat.
%5 nodig voor Faulty Process-eigenschap.

\bibliographystyle{plain}
\bibliography{main}

\end{document}
