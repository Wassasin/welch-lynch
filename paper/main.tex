\documentclass[a4paper,10pt]{article}
\usepackage[british]{babel}
\usepackage[T1]{fontenc}
\usepackage{hyperref}
\usepackage[a4paper]{geometry}

\usepackage{tabularx}
\usepackage{graphicx}
\usepackage{rotating}
\usepackage{float}
\usepackage{xspace}

% Math packages
\usepackage[sc]{mathpazo}
\usepackage{amsthm}
\usepackage{algorithmic}
\usepackage{amsmath}

\usepackage{listings}

% Document properties
\title{Welch-Lynch clock synchronization protocol \\\textit{Modelling in UPPAAL}}
\author{
	Wouter Geraedts \\ \small{\texttt{w.geraedts@student.ru.nl}} \and
	Ko Stoffelen     \\ \small{\texttt{kostoffelen@student.ru.nl}}
}
\date{}
\linespread{1.05}

% Terminology
\newcommand{\UPPAAL}{UPPAAL\xspace}

\begin{document}
	\maketitle

%%% Contents %%%
%Originele paper

%Zoeken naar concrete constanten

%Aanpassingen voor snelheid
%	Gebruik van C-functies [step1]
%	Gebruik van select-statements [step2]
%	Gebruik van local_time uit originele paper [step3]
%	Gebruik van SUM uit originele paper [step3minimized]
%	Timejumps [step3minimized]
%	Scalarset reduction [step4scalar]
%	Uitgebreide guards (forall i.p.v. local var C) [step2]

%Aanpassingen voor generiekerheid
%	Diff variable upgrade
%	Concrete types
%	Min en max (zelfs trager dan SUM) [step4]

%Resultaten: Vergelijkingen in performance
%Faulty process, niet mogelijk
%	Werk om te doen om het alsnog te maken (array voor min en max; zorgen dat het kan met scalarset (id_t))

\section{Introduction}

%Over dit paper
%Relatie met originele paper
%Opbouw van dit paper
%Korte verwijzing naar conclusie

\section{Optimizations}

%Aanpassingen voor snelheid
%Toelichten dat het leesbaarder is a.d.v. A First Introduction to UPPAAL

\section{Towards faulty process}

%Aanpassingen voor generiekerheid
%Aanpassingen die we nog wilde doen, indien we 5 processen hadden kunnen modelleren

\section{Remarks on Aceto2004Notes}

%Niet duidelijk of ze daadwerkelijk het model hebben kunnen doorrekenen
%Geen concrete waarden voor constanten; beschrijving van zoektocht
%Onduidelijk wat het type is van variablen

\section{Performance}

%Vergelijking tussen verschillende versies van ons, en uiteindelijke versie van aceto2004notes

\section{Conclusion}
	
%We kunnen succesvol 3 processen door laten rekenen, sneller dan toen (40 sec vs. <TODO>)
%Als 4 haalbaar blijkt te zijn, dan dat.
%5 nodig voor Faulty Process-eigenschap.

\end{document}
